\newglossaryentry{signature}{
    name=Pėdsakas,
    text=pėdsakas,
    description={Programos struktūros ir požymių santrauka, beveik unikaliai identifikuojanti programą (pvz., \gls{hashfunction})},
    user1=pėdsako,
    user2=pėdsakui,
    user3=pėdsaką,
    user4=pėdsaku,
    user5=pėdsake,
    plural=pėdsakai,
}

\newglossaryentry{hashfunction}{
    name=Maišymo funkcija,
    text=maišymo funkcija,
    description={Tai funkcija $f: \set{0,1}^* \rightarrow \set{0,1}^m$. Naudojama, kai iš begalinės įvesčių erdvės norima gauti fiksuoto dydžio ($m$) išvestį},
    user1=maišymo funkcijos,
    user2=maišymo funkcijai,
    user3=maišymo funkciją,
    user4=maišymo funkcija,
    user5=maišymo funkcijoje,
    plural=maišymo funkcijos,
}

\newglossaryentry{decisionBoundary}{
    name={Sprendimų priėmimo riba \angl{Decision Boundary}},
    description={Paprasčiausiems \gls{ml} modeliams tai yra kreivė plokštumoje. Sudėtingesniems -- daugiadimensiniams modeliams -- daugdara \angl{manifold}},
    text=sprendimų priėmimo riba,
    user1=sprendimų priėmimo ribos,
    user2=sprendimų priėmimo ribai,
    user3=sprendimų priėmimo ribą,
    user4=sprendimų priėmimo ribomis,
    user5=sprendimų priėmimo ribose,
    plural=sprendimų priėmimo ribos,
}

\newglossaryentry{adversarial}{
    name={Varžymosi principais pagrįstos atakos \angl{Adversarial Attacks}},
    description={Tai atakos, pritaikytos \enquote{apgauti} \gls{ml} klasifikatorius},
    text=varžymosi principais pagrįsta ataka,
    user1=varžymosi principais pagrįstos atakos,
    user2=varžymosi principais pagrįstai atakai,
    user3=varžymosi principais pagrįstą ataką,
    user4=varžymosi principais pagrįsta ataka,
    user5=varžymosi principais pagrįstoje atakoje,
    plural=varžymosi principais pagrįstos atakos,
}

% Acronyms
\newacronym{ml}{ML}{mašininis mokymasis}
\newacronym{di}{DI}{dirbtinis intelektas}
\newacronym{ae}{AE}{varžymosi principais pagrįstomis atakomis obfuskuoti kenkėjiško kodo pavyzdžiai}