\newglossaryentry{signature}{
    name=Pėdsakas,
    description={Programos struktūros ir požymių santrauka, beveik unikaliai identifikuojanti programą (pvz., \gls{hashfunction})},
    text=pėdsakas,
    ko=pėdsako,
    kam=pėdsakui,
    ka=pėdsaką,
    kuo=pėdsaku,
    kur=pėdsake,
    plural=pėdsakai,
    plko=pėdsakų,
    plkam=pėdsakams,
    plka=pėdsakus,
    plkuo=pėdsakais,
    plkur=pėdsakuose,
}

\newglossaryentry{hashfunction}{
    name=Maišymo funkcija,
    text=maišymo funkcija,
    description={Tai funkcija $f: \set{0,1}^* \rightarrow \set{0,1}^m$. Naudojama, kai iš begalinės įvesčių erdvės norima gauti fiksuoto dydžio ($m$) išvestį},
    ko=maišymo funkcijos,
    kam=maišymo funkcijai,
    ka=maišymo funkciją,
    kuo=maišymo funkcija,
    kur=maišymo funkcijoje,
    plural=maišymo funkcijos,
    plko=maišymo funkcijų,
    plkam=maišymo funkcijoms,
    plka=maišymo funkcijas,
    plkuo=maišymo funkcijomis,
    plkur=maišymo funkcijose,
}

\newglossaryentry{decisionBoundary}{
    name={Sprendimų priėmimo riba \angl{Decision Boundary}},
    description={Paprasčiausiems \gls{ml} modeliams tai yra kreivė plokštumoje. Sudėtingesniems -- daugiadimensiniams modeliams -- daugdara \angl{manifold}},
    text=sprendimų priėmimo riba,
    ko=sprendimų priėmimo ribos,
    kam=sprendimų priėmimo ribai,
    ka=sprendimų priėmimo ribą,
    kuo=sprendimų priėmimo ribomis,
    kur=sprendimų priėmimo ribose,
    plural=sprendimų priėmimo ribos,
    plko=sprendimų priėmimo ribos,
    plkam=sprendimų priėmimo riboms,
    plka=sprendimų priėmimo ribas,
    plkuo=sprendimų priėmimo ribomis,
    plkur=sprendimų priėmimo ribose,
}

\newglossaryentry{adversarial}{
    name={Varžymosi principais pagrįstos atakos \angl{Adversarial Attacks}},
    description={Tai atakos, pritaikytos „apgauti“ \gls{ml} klasifikatorius},
    text=varžymosi principais pagrįsta ataka,
    ko=varžymosi principais pagrįstos atakos,
    kam=varžymosi principais pagrįstai atakai,
    ka=varžymosi principais pagrįstą ataką,
    kuo=varžymosi principais pagrįsta ataka,
    kur=varžymosi principais pagrįstoje atakoje,
    plural=varžymosi principais pagrįstos atakos,
    plko=varžymosi principais pagrįstų atakų,
    plkam=varžymosi principais pagrįstoms atakoms,
    plka=varžymosi principais pagrįstas atakas,
    plkuo=varžymosi principais pagrįstomis atakomis,
    plkur=varžymosi principais pagrįstose atakose,
}

\newglossaryentry{framework}{
    name={Karkasas \angl{Framework}},
    description={Nurodo specifines technologijas, naudojamus požymius ir perturbacijas, siekiamus tikslus \gls{ae} generacijai. Skirtas apibrėžti procesą ir įrankius, kuriuos naudojant būtų galima generuoti nurodytų tikslų siekiančius \gls{ae}},
    text={karkasas},
    ko=karkaso,
    kam=karkasui,
    ka=karkasą,
    kuo=karkasu,
    kur=karkase,
    plural=karkasai,
    plko=karkasų,
    plkam=karkasams,
    plka=karkasus,
    plkuo=karkasais,
    plkur=karkasuose,
}

\newglossaryentry{zeroSumGame}{
    name={Nulinės sumos žaidimas \angl{Zero-Sum Game}},
    description={Dviejų žaidėjų žaidimas, kuriame galimas vienas laimėtojas. Laimėtojo laimėta suma yra lygi pralaimėtojo pralaimėtai sumai},
    text={nulinės sumos žaidimas},
    ko=nulinės sumos žaidimo,
    kam=nulinės sumos žaidimui,
    ka=nulinės sumos žaidimą,
    kuo=nulinės sumos žaidimu,
    kur=nulinės sumos žaidime,
    plural=nulinės sumos žaidimai,
    plko=nulinės sumos žaidimų,
    plkam=nulinės sumos žaidimams,
    plka=nulinės sumos žaidimus,
    plkuo=nulinės sumos žaidimais,
    plkur=nulinės sumos žaidimuose,
}

\newglossaryentry{surrogateModel}{
    name={Surogatinis Modelis \angl{Surrogate Model}},
    description={\gls{ml} modelis, aproksimuojantis kitą \gls{ml} modelį, kurio parametrai (svoriai) nėra žinomi},
    text={surogatinis modelis},
    ko=surogatinio modelio,
    kam=surogatiniam modeliui,
    ka=surogatinį modelį,
    kuo=surogatiniu modeliu,
    kur=surogatiniame modelyje,
    plural=surogatiniai modeliai,
    plko=surogatinių modelių,
    plkam=surogatiniams modeliams,
    plka=surogatinius modelius,
    plkuo=surogatiniais modeliais,
    plkur=surogatiniuose modeliuose,
}

\newglossaryentry{policy}{
    name={Strategija \angl{Policy}},
    description={Tai funkcija $\pi : S \times A \rightarrow \set{0,1}$, čia $S$ -- galimų būsenų erdvė \angl{State Space}, $A$ -- galimų veiksmų erdvė \angl{Action Space}. Šią funkciją \gls{rl} modelis „išmoksta“ mokymosi metu},
    text={strategija},
    ko=strategijos,
    kam=strategijai,
    ka=strategiją,
    kuo=strategija,
    kur=strategijoje,
    plural=strategijos,
    plko=strategijų,
    plkam=strategijoms,
    plka=strategijas,
    plkuo=strategijomis,
    plkur=strategijose,
}

\newglossaryentry{qfunction}{
    name={Q-Funkcija \angl{Q-Function}},
    description={$Q: S \times A \rightarrow \mathbb{R}$, čia $S$~--~galimų būsenų erdvė \angl{State Space}, $A$~--~galimų veiksmų erdvė \angl{Action Space}},
    text=$Q$-funkcija,
    ko=$Q$-funkcijos,
    kam=$Q$-funkcijai,
    ka=$Q$-funkciją,
    kuo=$Q$-funkcija,
    kur=$Q$-funkcijoje,
    plural=$Q$-funkcijos,
    plko=$Q$-funkcijų,
    plkam=$Q$-funkcijoms,
    plka=$Q$-funkcijas,
    plkuo=$Q$-funkcijomis,
    plkur=$Q$-funkcijose,
}

\newglossaryentry{adverasrialRetraining}{
    name=Varžymosi principais grįstas treniravimas,
    description={\gls{ml} modelio treniravimas naudojant \gls{ae} kaip mokymosi duomenis. Viena iš apsisaugojimo nuo \glsplko{adversarial} strategijų},
    text=varžymosi principais grįstas treniravimas,
    ko=varžymosi principais grįsto treniravimo,
    kam=varžymosi principais grįstam treniravimui,
    ka=varžymosi principais grįstą treniravimą,
    kuo=varžymosi principais grįstu treniravimu,
    kur=varžymosi principais grįstame treniravime,
    plural=varžymosi principais grįsti treniravimai,
    plko=varžymosi principais grįstų treniravimų,
    plkam=varžymosi principais grįstiems treniravimams,
    plka=varžymosi principais grįstus treniravimus,
    plkuo=varžymosi principais grįstais treniravimais,
    plkur=varžymosi principais grįstuose treniravimuose,
}

\newglossaryentry{blackBoxAttack}{
    name={„Juodos~dėžės“ ataka},
    description={\glsko{adversarial} atvejis, kai atakuojamo \gls{ml} modelio parametrai bei klasifikacijos tikimybiniai įverčiai nėra žinomi.},
    text={„juodos~dėžės“ ataka},
    ko={„juodos~dėžės“ atakos},
    kam={„juodos~dėžės“ atakai},
    ka={„juodos~dėžės“ ataką},
    kuo={„juodos~dėžės“ ataka},
    kur={„juodos~dėžės“ atakoje},
    plural={„juodos~dėžės“ atakos},
    plko={„juodos~dėžės“ atakų},
    plkam={„juodos~dėžės“ atakoms},
    plka={„juodos~dėžės“ atakas},
    plkuo={„juodos~dėžės“ atakomis},
    plkur={„juodos~dėžės“ atakose},
}

\newglossaryentry{inertia}{
    name={Inercija \angl{inertia}},
    description={variacijos dalis, kurią „paaiškina“ \gls{pca} komponentė},
    text={inercija},
    ko={inercijos},
    kam={inercijai},
    ka={inerciją},
    kuo={inercija},
    kur={inercijoje},
    plural={\errmessage{inertia in plural does not make sense}},
    plko={\errmessage{inertia in plural does not make sense}},
    plkam={\errmessage{inertia in plural does not make sense}},
    plka={\errmessage{inertia in plural does not make sense}},
    plkuo={\errmessage{inertia in plural does not make sense}},
    plkur={\errmessage{inertia in plural does not make sense}},
}

% Acronyms
\newacronym{ml}{ML}{Mašininis mokymasis \angl{machine learning}}
\newacronym{di}{DI}{Dirbtinis intelektas \angl{artificial intelligence}}
\newacronym{ae}{AE}{Varžymosi principais pagrįstomis atakomis obfuskuoti kenkėjiško kodo pavyzdžiai \angl{adversarial examples}}
\newacronym{pe}{PE}{\Angl{portable executable}}
\newacronym{dll}{DLL}{Dinamiškai susieta biblioteka \angl{dynamic link library}}
\newacronym{api}{API}{\Angl{Application Programming Interface}}
\newacronym{nlp}{NLP}{Skaitmeninis natūraliosios kalbos apdorojimas \angl{natural language processing}}
\newacronym{gan}{GAN}{Generatyviniai priešiški tinklai \angl{generative adversarial networks}}
\newacronym{svm}{SVM}{\Angl{support vector machine}}
\newacronym{knn}{KNN}{\Angl{$K$-Nearest Neighbours}}
\newacronym{gbdt}{GBDT}{\Angl{gradient boosted decision trees}}
\newacronym{cnn}{CNN}{\Angl{convolutional neural network}}
\newacronym{rl}{RL}{Skatinamasis mokymasis \angl{reinforcement learning}}
\newacronym{genetic}{GA}{Genetiniais algoritmais pagrįstas \gls{ml} modelis \angl{genetic algorithm}}

\newacronym{pca}{PCA}{\Angl{Principal Component Analysis}}
\newacronym{mca}{MCA}{\Angl{Multiple Correspondence Analysis}}
\newacronym{ids}{IDS}{\Angl{Intrusion Detection System}}
\newacronym{xai}{XAI}{paaiškinamas dirbtinis intelektas \angl{explainable artificial intelligence}}
\newacronym{lime}{LIME}{\Angl{Local Interpretable Model-agnostic Explanations} -- lokalūs, interpretuojami \gls{ml} modelių išvesčių paaiškinimai.}