\subsection{Skatinamojo mokymosi tipo modelių \glspl{framework}}\label{sec:literature:rl}

Skatinamojo mokymosi (\gls{rl}) modeliai susideda iš agento ir aplinkos.
Aplinka susideda iš informatyvių požymių ištraukimo metodo (\textit{angl.
    feature extraction}) ir kenkėjiškų programų detektoriaus. Šiuo atveju aplinkos
būsenų erdvė $S$ yra požymių vektorių erdvė. Agentas -- tai algoritmas ar
neuroninis tinklas, kurio tikslas yra surasti optimalią \glska{policy}. Šiuo atveju strategijos veiksmų erdvė $A$ susideda iš
perturbacijų \skyrius{sec:literature:perturbations}
\cite{fangEvadingMalwareEngines2019}. Bendras \gls{rl} modelių mokymosi etapas
yra tokia seka \cite{fangEvadingMalwareEngines2019,
    zhongReinforcementLearningBased2022}:
\begin{enumerate}
    \item Agentas, naudodamas dabartinę aplinkos būseną ir praeito veiksmo atlygį
          \angl{reward}, parenka sekantį veiksmą iš galimų veiksmų aibės ir taiko
          mokymosi algoritmą (algoritmas priklauso nuo agento įgyvendinimo).
    \item Atliekamas veiksmas -- perturbuojama programa arba požymių vektorius (priklauso
          nuo \glsko{framework}).
    \item Gaunami aplinkos kitimo įverčiai -- nauja būsena ir atlygis, skaičiuojamas
          pagal detektoriaus klasifikacijos rezultatą.
    \item Seka kartojama tol, kol agentas nelaiko strategijos optimalia arba nustatytą
          kiekį kartų.
\end{enumerate}

\begin{describeFramework}{DQEAF}{\cite{fangEvadingMalwareEngines2019}}
    \introLastPar{
        Šis \gls{framework} taiko gilųjį skatinamąjį mokymąsi, kai agentas įgyvendinamas kaip gilusis neuroninis tinklas.
    }
    \purpose{Parodyti, jog \gls{ml} kenkėjiškų programų detektoriai, ypač modeliai, išmokyti prižiūrimu mokymusi, yra pažeidžiami \glsplkam{adversarial}.
    }
    \surrogate{\gls{rl} karkasuose nenaudojami surogatiniai modeliai. Kaip \enquote{juodos dėžės} detektorius pasirinktas \gls{gbdt} modelis.}
    \mainModel{Agentas įgyvendintas kaip gilusis $Q$-tinklas (\gls{cnn} praplėtimas, kai tinklas naudojamas kaip \glsko{qfunction} aproksimacija). Taip pat taikomas prioritetizuotas patirčių pakartojimo metodas \angl{prioritized experience replay}, kuomet agentas mokomas tik su aukštą atlygį gavusiais perėjimais ($S \times A$).}
    \features{Požymių vektorius taip pat apibrėžia visų būsenų erdvę $S$. Šiuo atveju $S = \mathbb{R}^{513}$.}{
        Baitų lygio požymiai \skyrius{sec:literature:features:byte} -- baitų/entropijos
        histograma. 
    }

    \perturbations{Perturbacijos apibrėžia visų galimų agento veiksmų
        erdvę $A$. Šiuo atveju $A = \set{0,1}^4$.}{
        Baitų lygio perturbacijos \skyrius{sec:literature:perturbations:byte}:
        \begin{itemize}
            \item \textit{ARBE}
            \item \textit{ARI}
            \item \textit{ARS}
            \item \textit{RS}
        \end{itemize}
    }
\end{describeFramework}

\begin{describeFramework}{MalInfo}{\cite{zhongReinforcementLearningBased2022}}
    \introLastPar{
        \textit{\Name} remiasi \refFramework{MalFox}.
    }
    \purpose{Surasti optimalią obfuskacijos strategiją konkrečiai programai, pagal kurią sukurtas \gls{ae} nebūtų aptiktas komercinių kenkėjiškų programų detektorių.
    }
    \surrogate{\gls{rl} surogatinis modelis nenaudojamas. \enquote{Juodos dėžės} detektoriumi pasirinkti komerciniai detektoriai (\textit{VirusTotal}).
    }
    \mainModel{Agentas įgyvendintas kaip klasikiniai \gls{ml} algoritmai (konkrečiai dinaminis programavimas ir skirtumų laike \angl{temporal difference} algoritmas).
    }
    \features{Agentas nėra neuroninis tinklas ir požymių iš programos netraukia. Agentas mokosi tik iš perėjimų, o būsenų erdvė $S$ yra originali programa ir perturbuoti jos variantai. Teoriškai perturbuotų programos variantų galėtų būti be galo daug, tuomet $S = A^\infty, |S| = \aleph_0$, tačiau autoriai nurodo, jog daugiau nei 3 sluoksniai kompleksinių perturbacijų reikšmingai paveikia programos veikimo laiką, o tai gali \enquote{sukelti įtarimų} komerciniams detektoriams. Todėl pasirinkta $S = A^3$.
    }{}
    \perturbations{$A = \set{0,1,2,3}$}{
        \begin{itemize}
            \item \textit{Null} perturbacija -- naudinga tik formaliam pilnumui (atitinka nulinį $A$ veiksmą).
            \item Visos kompleksinės perturbacijos \skyrius{sec:literature:perturbations:complex},
                  t.~y. tokios pačios, kaip ir \refFramework{MalFox} \glsko{framework}. 
        \end{itemize}
    }
\end{describeFramework}