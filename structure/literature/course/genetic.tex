\subsection{Genetinių algoritmų tipo modelių \glspl{framework}}\label{sec:literature:genetic}

Genetinai algoritmai (\gls{genetic}) yra viena seniausių mašininio mokymosi
\gls{ml} apraiškų; jų veikimas paremtas evoliucija
\cite{castroAIMEDEvolvingMalware2019}. Kenkėjiškų programų obfuskacijai
\gls{ae} generavimas taikant \gls{genetic} yra tokia seka
\cite{yusteOptimizationCodeCaves2022}:
\begin{enumerate}
    \item Sukuriama pradinė populiacija (perturbacijos metodai pradinei populiacijai
          priklauso nuo \glsko{framework}).
    \item Atliekamas tinkamumo \angl{fitness} vertinimas.\label{enum:genetic:fitness}
    \item Atliekama selekcija -- dažniausiai pasirenkami geriausiai įvertinti
          populiacijos \gls{ae}, tačiau galimos ir kitos selekcijos strategijos.
    \item Atliekamas selekcijos atrinktų \gls{ae} kryžminimas (po 2) taip sukuriant naują
          \gls{ae}, turintį po dalį genų iš abiejų kryžmintų \gls{ae}.
    \item Tam tikrai daliai \gls{ae} atliekama dalies genų mutacija.
    \item Vertinama, ar sugeneruoti \gls{ae} atitinka kriterijus (vertina detektorius).
    \item Jei kriterijai nėra tenkinami, seka kartojama nuo \ref{enum:genetic:fitness}-o
          žingsnio.
\end{enumerate}

\begin{describeFramework}{AIMED}{\cite{castroAIMEDEvolvingMalware2019}}
    \purpose{\gls{ae} generavimo greičio padidinimas ir modelių kompleksiškumo sumažinimas, lyginant su \gls{gan} ir \gls{rl} tipo modelių karkasais.}
    \surrogate{Surogatinis modelis nenaudojamas. Naudojami \enquote{juodos dėžės} detektoriai yra \textbf{3 komerciniai} (\textit{Kaspersky, ESET, Sophos}) ir vienas \gls{ml} modelis -- \gls{gbdt}.}
    \mainModel{Klasikinis \gls{genetic} modelis -- veikimas visiškai atitinką bendrą seką. Tinkamumo \angl{fitness} vertinamas remiasi \gls{ae} požymių vektoriaus panašumu į originalios programos požymių vektorių (kuo mažiau panašūs, tuo tinkamumo įvertinimas didesnis).}
    \features{}{
        Baitų lygio požymiai \skyrius{sec:literature:features:byte} -- atskiras $n$-gramų
        atvejis, kai $n=1$.
    }
    \perturbations{}{ Baitų lygio
        perturbacijos\footnote{autoriai rėmėsi perturbacijomis, aprašytomis
            \cite{andersonLearningEvadeStatic2018}}
        \skyrius{sec:literature:perturbations:byte}. 
    }
\end{describeFramework}

\clearpage
\begin{describeFramework}{GAMMA}{\cite{demetrioFunctionalityPreservingBlackBoxOptimization2021}}
    \purpose{Efektyvus (neaptikimo šansų didinimas naudojant perturbacijas, paremtas nekenksmingomis programomis) \glsplko{adversarial} kūrimas.}
    \surrogate{Surogatinis modelis nenaudojamas. \gls{gbdt} ir \textit{MalConv} pasirinkti kaip \enquote{juodos dėžės} detektoriai.}
    \mainModel{Pagrindinė modelio idėja yra požymių ištraukimas iš nekenksmingų programų ir jų pridėjimas, naudojant tam pritaikytas perturbacijas, į kenksmingas programas kiekvienos populiacijos generavimo metu. Tinkamumo \angl{fitness} ir kriterijų vertinimas atliekamas naudojant detektorių ir pridėtų požymių dydį baitais (norima pridėti kuo mažiau požymių).}
    \features{}{
        \begin{itemize}
            \item \gls{pe} formato programų požymiai \skyrius{sec:literature:features:pe}.
            \item Kodas sekcijose (nestandartinis požymis).
        \end{itemize}
    }
    \perturbations{}{
        \begin{itemize}
            \item Visos baitų lygio perturbacijos \skyrius{sec:literature:perturbations:byte},
                  gebančios pridėti baitus.
            \item Autoriai pažymi, jog gali būti naudojama ir \gls{dll} / \gls{api} vardų
                  pridėjimo semantinė perturbacija \skyrius{sec:literature:perturbations:semantic}.
        \end{itemize}
    }
\end{describeFramework}