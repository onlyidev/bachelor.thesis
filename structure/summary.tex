\section*{Santrauka}

Šiame darbe nagrinėjamos \glspl{adversarial} prieš kenkėjiškų programų detektorius bei tokių atakų aptikimo ir apsisaugojimo nuo jų strategijos. Siekiant patobulinti jau esamus aptikimo metodus bei pritaikyti juos bet kokiems požymių vektoriams, siūloma sujungti \textit{\gls{mca}} dimensijų mažinimo metodą ir \textit{\LIME} -- \gls{ml} modelių sprendimų paaiškinimo metodą, kurie atitinka mokslinėje literatūroje minimas perspektyviausias \glsplko{adversarial} aptikimo strategijas. Atliekamas siūlomo metodo tyrimas nustatant jo tikslumą bei lyginant su prieš tai minėtomis \glsplko{adversarial} aptikimo strategijomis. Tyrimui pasitelkiamas \textit{MalGAN} karkasas kaip tokių atakų generatorius, kadangi šio karkaso naudojami dvejetainiai požymių vektoriai turėtų kelti daugiausia sunkumų esamiems apsisaugojimo nuo \glsplko{adversarial} metodams. Nustatyta, jog pasiūlytas metodas geba tiek apsaugoti nuo, tiek aptikti \glsplka{adversarial} efektyviau nei esami metodai. 

\clearpage
\section*{Summary}

This work analyses \glsen{adversarial}{adversarial attacks} against malware detectors as well as strategies of detection and defense against such attacks. With the goal to improve already existing detection methods and adapt them for use with any feature vectors, this work explores combining \textit{\gls{mca}} dimensionality reduction method and \textit{\LIME} -- a method for explaining \gls{ml} models' decisions as these are implementations of the most perspective strategies of defense against adversarial attacks mentioned in scientific literature. A study is conducted to determine the accuracy of the suggested method by comparing it with the aforementioned strategies of defense against \glsen{adversarial}{adversarial attacks}. \textit{MalGAN} framework is used for this study as the generator of such attacks since the binary vectors used by it should cause the most difficulty for existing strategies of defense. Experimental results show that the suggested method is able to both defend against and detect \glsen{adversarial}{adversarial attacks} more effectively and consistently than currently used methods.