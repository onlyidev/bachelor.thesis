\section{Rezultatai ir išvados}

\subsection*{Rezultatai}

\begin{table}[h]
    \centering
    \caption{Eksperimentų rezultatų (tikslumo metrikų) suvestinė}
    \begin{tabular}{l|c|c}
        \bfseries Eksperimentas &\bfseries Tikslumas (K/N)\footnotemark &\bfseries Tikslumas (K/N/O)\footnotemark \\ \hline
        \ref{sec:exp:1} Bazinis & \num{\accNormal} & - \\
        \ref{sec:exp:4} \gls{mca} & \num{\accMcaEquiv} & - \\
        \ref{sec:exp:2} \LIME & \num{\accLimeCat} & \num{\accLimeCatObf} \\
        \ref{sec:exp:3} \LIME + \gls{mca} & \num{\accLime} & \num{\accLimeObf} \\
    \end{tabular}
    \label{tbl:exp:summary}
\end{table}
\addtocounter{footnote}{-1}
\footnotetext{Kenkėjiška / Nekenkėjiška}
\stepcounter{footnote}
\footnotetext{Kenkėjiška / Nekenkėjiška / Obfuskuota}

\begin{enumerate}
    \item Apžvelgti ir pritaikyti mokslinėje literatūroje minimi kodo obfuskacijos bei \glsplko{adversarial} -- \gls{ae} generavimo -- metodai, jų aptikimo strategijos \skyrius{sec:literature}.
    \item Pasiūlytas naujas \gls{ae} aptikimo metodas, pritaikomas bet kokiam kenkėjiškų programų detektoriui bei gebantis apdoroti dvejetainius požymius, apjungiantis \LIME pritaikymo \gls{ae} aptikimui ir \gls{mca} transformacijos idėjas \skyrius{sec:method}.
    \item Atliktas \gls{ae} aptikimo metodų lyginamosios analizės tyrimas \skyrius{sec:experiments} ir nustatytas pasiūlyto \gls{ae} aptikimo metodo \skyrius{sec:method} efektyvumas (matuojant modelio tikslumą ir atsižvelgiant į kitas klasifikacijos metrikas), lyginant pasiūlytą metodą su jo sudedamosiomis dalimis \Zr{tbl:exp:summary}.
\end{enumerate}


\subsection*{Išvados}

\begin{enumerate}
    \item Autoriaus siūlomas \gls{ae} aptikimo metodas lyginamosios analizės tyrime pasiekė geriausią rezultatą (didžiausią tikslumą). Taip pat buvo pranašesnis už likusius modelius visomis kitomis lyginamomis metrikomis (preciziškumu, atkūrimu ir F1).
    \item \TODO
\end{enumerate}