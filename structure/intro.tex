\sectionnonum{Įvadas}

Pastaraisiais metais kenkėjiškas kodas ir programos kuriamos itin sparčiai (\sim450000 kenkėjiškų programų per dieną \href{https://www.av-test.org/en/statistics/malware/}{2024 m. AV-TEST}\footnote{https://www.av-test.org/en/statistics/malware} duomenimis). Kenkėjiško kodo aptikimo programos, kurios tradiciškai remiasi programų \glsplkuo{signature}, nespėja atnaujinti pėdsakų duomenų bazių pakankamai greitai. Dėl to \gls{di}, tiksliau mašininio mokymosi (\gls{ml}), naudojimas kenkėjiškų programų ar kenkėjiško kodo aptikimo srityje tapo itin populiarus \cite{demetrioAdversarialEXEmplesSurvey2021}. Tačiau \gls{ml} modeliai, nors ir geba aptikti kenkėjiškas programas iš naujų, dar nematytų, duomenų, yra pažeidžiami \glsplkam{adversarial} \cite{castroAIMEDEvolvingMalware2019,huGeneratingAdversarialMalware2017,rosenbergGenericBlackBoxEndEnd2018,zhongReinforcementLearningBased2022}. Šių atakų principas yra \gls{ml} modelio --~klasifikatoriaus~-- \glsko{decisionBoundary} radimas -- žinant šią ribą pakanka pakeisti kenkėjiškos programos veikimą taip, kad \gls{ml} modelis priimtų sprendimą klasifikuoti ją kaip nekenksmingą \cite{demetrioAdversarialEXEmplesSurvey2021}. Nustatyta, jog šią ribą galima rasti tiek žinant klasifikatoriaus parametrus, tiek jų nežinant ir net turint labai ribotą prieigą prie klasifikatoriaus rezultatų (pvz., klasifikacijos rezultatą be tikimybių -- tokios sąlygos vadinamos \enquote{juodos dėžės} atvejai) \cite{fangEvadingMalwareEngines2019}.

Vis tik \glspl{adversarial} nėra neįveikiamos. Nuolat kuriami nauji jų aptikimo metodai, tokie kaip \gls{adverasrialRetraining}, gradientų slėpimas ir kt. Kiekvienas metodas turi savų stiprybių ir silpnybių bei dažniausiai remiasi viena iš specifinio \gls{ml} modelio įgyvendinimo savybių, kitaip tariant, nėra vieno geriausio, tinkamiausio ar teoriškai teisingo \glsplko{adversarial} aptikimo metodo. 
Tiksliau, nėra pačių \gls{ae} konstravimo teorinio modelio, dėl šio proceso kompleksiškumo, tad jų aptikimo strategijos teorinis modelis taip pat nėra žinomas \cite{chakrabortySurveyAdversarialAttacks2021}. Šiame darbe siekiama generalizuoti \gls{ae} aptikimą apjungiant panašiame kenkėjiškų programų aptikimo kontekste naudojamą \LIME \cite{ribeiroWhyShouldTrust2016} metodą ir kitas mokslinėje literatūroje aprašytas technikas.

\vspace{10pt}
\textbf{Tikslas} -- pritaikyti \LIME metodą sėkmingų varžymosi principais grįstų atakų aptikimui prieš kenkėjiškų programų detektorius vertinant bet kokius požymius.

\vspace{10pt}
\textbf{Uždaviniai}:
\begin{enumerate}
    \item Apžvelgti kenkėjiško kodo obfuskacijos metodus bei apsisaugojimo nuo jų strategijas.
    \item Pritaikyti \glsplko{adversarial} aptikimą dvejetainius požymių vektorius naudojantiems modeliams taikant dimensijų mažinimo metodus.
    \item Sukurti klasifikavimo proceso praplėtimą į jį įtraukiant \glspl{adversarial} aptikimą.
    \item Ištirti praplėsto klasifikavimo proceso tikslumą \angl{accuracy}.
\end{enumerate}