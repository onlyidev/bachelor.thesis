\section{Darbo planas}

\subsection{Tyrimo objektas ir aktualumas}

Pagrindinis tyrimo objektas -- varžymosi principais grįstos atakos prieš
kenkėjiškų programų detektorius. Varžymosi principais grįstos atakos -- tai subtilus duomenų modifikacijos metodas. Šių atakų metu duomenys pakeičiami nežymiai, tačiau tokiu būdu, jog geba suklaidinti jiems pritaikytus dirbtiniu intelektu įgyvendintus klasifikatorius. Vienas iš tokių atakų panaudojimo atvejų -- kenkėjiškų programų obfuskacija -- procesas, kurio metu kenkėjiškos programos pakeičiamos taip, jog jų veikimas išlieka toks pat, tačiau kenkėjiškų programų detektoriai šių modifikuotų programų neaptinka (klasifikuoja kaip nekenksmingas). Kadangi dauguma kenkėjiškų programų detektorių yra įgyvendinti kaip dirbtinio intelekto algoritmai ar modeliai, svarbu ieškoti būdu aptikti tokias atakas. Vienas iš galimų ir mokslinėje literatūroje minimų būdų varžymosi principais grįstų atakų aptikimui -- dirbtinio intelekto modelio priimtų sprendimų paaiškinimas ir jo analizė. Kadangi šiuolaikiniai dirbtinio intelekto modeliai yra sudėtingi, daugiadimensiai ir sunkiai ar visiškai neinterpretuojami, modelio sprendimų priėmimo paaiškinimas suprantamas kaip paprastesnio (klasikinio statistinio) modelio ar metodo panaudojimas suprantamesnei atskiro atvejo klasifikacijos ir klasifikavimo proceso interpretacijai.

\subsection{Darbo tikslas}

Sukurti metodą sėkmingų varžymosi principais grįstų atakų aptikimui ir
paaiškinimui.

\subsection{Keliami uždaviniai ir laukiami rezultatai}

\subsubsection{Uždaviniai}
\begin{enumerate}
    \item Apibrėžti varžymosi principais grįstų atakų aptikimo metodą naudojant LIME (\textit{angl. Local Interpretable Model-Agnostic Explanations}).
    \item Pritaikyti varžymosi principais grįstų atakų aptikimą dvejetainius požymių vektorius naudojantiems modeliams taikant dimensijų redukavimo metodus.
    \item Sukurti klasifikavimo proceso praplėtimą į jį įtraukiant varžymosi principais
          grįstų atakų aptikimą.
    \item Ištirti praplėsto klasifikavimo proceso efektyvumą.
\end{enumerate}

\subsubsection{Laukiami rezultatai}
\begin{enumerate}
    \item Sukurtas varžymosi principais pagrįstų atakų aptikimo metodas, pritaikomas bet kokiam kenkėjiškų programų detektoriui.
    \item Laikantis klasifikavimo proceso praplėtimo aptinkama daugiau obfuskuotų kenkėjiškų programų, nei jo nesilaikant.
    \item Klasifikavimo proceso praplėtimas atskleidžia atakų vektorius.
\end{enumerate}

\subsection{Tyrimo metodai}

\begin{enumerate}
    \item \textbf{Modeliavimo metodas}. Klasifikavimo proceso praplėtimo tyrime bus naudojami šie modeliai:
    \begin{itemize}
        \item kenkėjiškų programų klasifikatoriaus modelis, įgyvendintas kaip mašininio mokymosi algoritmas,
        \item varžymosi principais grįstų atakų generavimo modelis, įgyvendintas kaip \textit{MalGAN} karkasą atitinkantis mašinino mokymosi modelis.
    \end{itemize}
    \item \textbf{Lyginamosios analizės metodas}. Lyginami klasifikavimo (kenkėjiškų programų aptikimo) rezultatai taikant klasifikavimo proceso praplėtimą ir jo netaikant.
\end{enumerate}

\subsection{Numatomas darbo atlikimo procesas}

\begin{enumerate}
    \item Literatūros analizė.
    \item Prieiga prie kenkėjiškų ir nekenkėjiškų programų požymių duomenų rinkinio \textit{SLEIPNIR}.
    \item \textit{MalGAN} karkaso ir kenkėjiškų programų detektoriaus modelių mokymas.
    \item \textit{LIME} komponento paruošimas varžymosi principais grįstų atakų aptikimui ir paaiškinimui (\textit{normalių požymių} aibės kūrimas iš mokymosi duomenų rinkinio).
    \item Dvejetainių požymių vektorių supaprastinimas redukuojant dimensijas (naudojant \textit{Multiple Correspondence Analysis} mokymosi duomenims) ir šio komponento integracija su \textit{LIME} komponentu.
    \item Klasifikavimo proceso praplėtimo apibrėžimas ir iliustracijų rengimas.
    \item Klasifikavimo proceso praplėtimo efektyvumo tyrimas taikant lyginamąją analizę.
\end{enumerate}