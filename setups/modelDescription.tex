\newenvironment{describeFramework}[2]{
    \newcommand{\purpose}[1]{\gdef\Purpose{##1}}
    \newcommand{\surrogate}[1]{\gdef\Surrogate{##1}}
    \newcommand{\mainModel}[1]{\gdef\MainModel{##1}}
    \newcommand{\features}[2]{\gdef\PreFeatures{##1} \gdef\Features{##2}}
    \newcommand{\perturbations}[2]{\gdef\PrePerturbations{##1} \gdef\Perturbations{##2}}
    \newcommand{\introLastPar}[1]{\gdef\IntroLastPar{##1}}

    \def\Purpose{}
    \def\Surrogate{}
    \def\MainModel{}
    \def\IntroLastPar{}
    \def\PreFeatures{}
    \def\Features{\item[]}
    \def\PrePerturbations{}
    \def\Perturbations{\item[]}
    \def\Name{#1}
    \def\Citation{#2}
    \subsubsection{\textit{\Name}}\label{sec:literature:framework:\Name}
}{
    \IntroLastPar{}\textit{\Name} \glsko{framework} \Citation{} tikslas ir apibrėžimas pateikiami \ref{tab:framework:\Name}-oje lentelėje.
    \begin{longtable}{p{0.15\textwidth - 0.675cm}|p{0.85\textwidth}}
        \caption{\textit{\Name} \gls{framework}}\label{tab:framework:\Name} \\
        \textbf{Tikslas}              & \Purpose{}                               \\ \toprule
        \textbf{\Gls{surrogateModel}} & \Surrogate{}                             \\ \midrule
        \textbf{\gls{ml} modelis}     & \MainModel{}                             \\ \midrule
        \textbf{Požymiai}             & \PreFeatures{}\Features{}                \\ \midrule
        \textbf{Perturbacijos}        & \PrePerturbations{}\Perturbations{}      \\ \bottomrule
    \end{longtable}
    \noindent
}

\newcommand{\refFramework}[1]{\textit{#1} (\ref{sec:literature:framework:#1})}